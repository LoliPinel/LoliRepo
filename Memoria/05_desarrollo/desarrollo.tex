\chapter{Control Architecture}
\section{Simple Inverted Pendulum Mode}
The simple inverted pendulum is the most basic model used to simplify humanoids' body. The basis of the pendulum are a mass $m$ linked to a pivot point $0$ by means of a massless link of longitude $l$ as in Figure ???.

The mass $m$ represents the total mass of the modelled system, a humanoid robot in this case, located at its Centre of Mass (CoM), and the longitude $l$ is the distance between the pivot point to the CoM. If gravitational force ($F_g$) is considered the only force acting in the system, the movement of the inverted pendulum is:
\begin{equation}
\sum{\Gamma_0} = ml^2 \ddot{\theta} - mgl\sin\theta
\label{eq:pendulo}
\end{equation}

It is assumed that angles are small enough and the $\sin\theta = 1$. Then, equation \ref{eq:pendulo} changes to:
\begin{equation}
\sum{\Gamma_0} = ml^2 \ddot{\theta} - mgl\theta
\label{eq:pendulo2}
\end{equation}

Also, the relation between the .................... and stiffness $k$ gives:
\begin{equation}
\Gamma = k(u-\theta)
\label{eq:pendulo3}
\end{equation}

The transfer functions of equations \ref{eq:pendulo2} and \ref{eq:pendulo3} are:
\begin{equation}
\Gamma(s) = mls^2\theta(s) - mgl\theta(s)
\end{equation}

\begin{equation}
\Gamma(s) = k(U(s)-\theta(s))
\end{equation}

And the transfer function that provides $\Gamma(s)$ as output getting $U(s)$ as input is:
\begin{equation}
\frac{\Gamma(s)}{U(s)} = -k \frac{s^2+(\beta - \alpha)}{s^2 + \alpha}
\label{eq:TFpar}
\end{equation}
where:
\begin{equation}
\alpha = \frac{k-mgl}{ml^2}
\end{equation}
\begin{equation}
\beta = \frac{k}{ml^2}
\end{equation}

From the relation between COG and ZMP described in chapter ????, in a general case we have:
\begin{equation}
\Gamma(s) = -P_{ZMP}(s) \cdot F_z
\label{eq:TFzmp}
\end{equation}

Then, from equations \ref{eq:TFpar} and \ref{eq:TFzmp} is is obtained the transfer function between the Torque and ZMP:
\begin{equation}
\frac{P_{ZMP}(s)}{U(s)} = -k \frac{s^2+(\beta - \alpha)}{s^2 + \alpha}
\end{equation}

From transfer function equation, we can obtain the space state model equations:
\begin{equation}
\begin{bmatrix}
\dot{x_1} \\
\dot{x_2}
\end{bmatrix} 
= 
\begin{bmatrix}
0 & 1 \\
-\alpha & 0
\end{bmatrix}
\begin{bmatrix}
x_1 \\
x_2
\end{bmatrix}
+
\begin{bmatrix}
0 \\
1
\end{bmatrix}
u
\end{equation}
\begin{equation}
y = \begin{bmatrix}
k\beta & 0 
\end{bmatrix}
+ \begin{bmatrix}
-k
\end{bmatrix}
u
\end{equation}

where $\dot{x_1} $ and $\dot{x_2}$ are the $x_{ZMP}$ and $y_{ZMP}$ respectively.


\section{Feedback in state space. The Linear Quadratic Regulator}

The quadratic optimal control method is one of the control methods applied in state space systems and it provides a systematic way of computing the state feedback control gain matrix [Referencia OGATA].
Given the state space system equation
\begin{equation}
\dot{x} = Ax+Bu ,
\label{eq:sseq}
\end{equation}
the LQR determines the matrix $K$ of the optima control vector
\begin{equation}
u(t) = -Kx(t)
\label{eq:control}
\end{equation}
so as to minimize the performance index
\begin{equation}
J = \int_{0}^{\infty}(x^{T}Qx+u^{T}Ru) dt
\end{equation}

where $Q$ is a positive-definite (or positive-semidefinite) Hermitian or real symmetric matrix and $R$ is a positive-definite Hermitian or real symmetric matrix. Note that the matrices $Q$ and $R$ determine the relative importance of the error and the expenditure of the energy of the control signals.
The linear control law given by equation \ref{eq:control} is the optimal control law. Therefore, if the unknown elements of the matrix $K = [K_1 \quad K_2]$ are determined so as to minimize the performance index, then $u(t) = -Kx$  is optimal for any initial state $x(0)$. The block diagram showing the optimal configuration for the single inverted pendulum system is presented in Figure ??? [DIAGRAMA DE BLOQUES. MIRAR TESIS DMITRY]. The controller maintain desired ZMP positions ($x_{ZMP}$, $y_{ZMP}$) of the single pendulum close to zero. Thus, the reference input of the control system in Figure [REF] is zero. Note that the ZMP positions $x_{ZMP}$ and $y_{ZMP}$ are $x_1$ and $x_2$, respectively, and the torque $\tau$ is represented by output $y$ in the diagram.

The optimum $K$ matrix is obtained from equations \ref{eq:Kcont} and \ref{eq:Kdisc}.
\begin{equation}
K = R^{-1}B^{T}P \quad (continuous \quad case)
\label{eq:Kcont}
\end{equation}
\begin{equation}
K = (R + B^{T}PB)^{-1}B^{T}PA \quad (discrete \quad case)
\label{eq:Kdisc}
\end{equation}

where $P$ is a positive-definite Hermitian or real symmetric matrix and it is necessary to compute the algebraic Ricatti Equation
\begin{equation}
P \rightarrow A^{T}P+PA-PBR^{-1}B^{T}P+Q = 0 \quad (continuous \quad case)
\end{equation}
\begin{equation}
P \rightarrow A^{T}PA+P-A^{T}PB(R+B^{T}PB)^{-1}B^{T}PA+Q = 0 \quad (discrete \quad case)
\end{equation}


In order to obtain the controller design for further simulations and experiments, the following mechanical parameters of the inverted pendulum (corresponding to Rh-2 humanoid robot) were taken: $m = 62.416 kg$, $l=0.8449 m$

Valor de stiffness???

For the optimum response of the control system, we took $Q = C^{T}C = \begin{bmatrix}
5.037 \cdot 10^{-8} & 0\\
0 & 0
\end{bmatrix}$ and $R = [1] $. After the LQR controller was designed, the control gains matrix $K = [23.1777 \quad 6.8027]$ was obtained









